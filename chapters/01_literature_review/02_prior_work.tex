\section{Prior Work}

Here is an example of a more complex image embedding, where you can refer to the entire figure \ref{fig:parallelismtypes} or even to one of its subfigures \ref{fig:modelparallel}.

\begin{figure}[ht]
	\centering
	\begin{subfigure}[b]{0.3\textwidth}
		\centering
		\includegraphics[width=\textwidth]{Dataparallel.png}
		\caption{Data parallelism}
		\label{fig:dataparallel}
	\end{subfigure}
	\hfill
	\begin{subfigure}[b]{0.3\textwidth}
		\centering
		\includegraphics[width=\textwidth]{Modelparallel.png}
		\caption{Model parallelism}
		\label{fig:modelparallel}
	\end{subfigure}
	\hfill
	\begin{subfigure}[b]{0.3\textwidth}
		\centering
		\includegraphics[width=\textwidth]{Layerparallel.png}
		\caption{Pipeline parallelism}
		\label{fig:pipelineparallel}
	\end{subfigure}
	\caption{Types of parallelism}
	\label{fig:parallelismtypes}
\end{figure}



There is also an example of a table in \ref{tab:phi_def}.

\begin{table}[h!]
\centering
\begin{tabular}{|c|l|}
\hline
\textbf{Type}                & \textbf{Definition of \(\phi\)} \\ \hline
Simple graph                 & \(\phi : E \to \mathcal{P}_2(V)\), where \(\mathcal{P}_2(V)\) is the set of 2-element subsets of \(V\) \\ \hline
Directed graph               & \(\phi : E \to V \times V\), where \((u, v)\) is an ordered pair \\ \hline
Multigraph                   & \(\phi : E \to \mathcal{P}_2(V) \times \mathbb{N}\), where \(\mathbb{N}\) is the index set of edges \\ \hline
Hypergraph                   & \(\phi : E \to \mathcal{P}(V)\), where \(\mathcal{P}(V)\) is the set of all subsets of \(V\) \\ \hline
\end{tabular}
\caption{Different definitions of the mapping \(\phi\) for various graph types.}
\label{tab:phi_def}
\end{table}

\lipsum[2-5]